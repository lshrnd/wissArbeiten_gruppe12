\documentclass[paper=a4,                 % Papierformat
               fontsize=12pt,            % (Standard-) Schriftgröße
               parskip=half,             % Absatzformat halbzeilig
               ngerman,                 % deutsche Trennung
               ]{scrartcl}

\usepackage[T1]{fontenc}                 % Schriftsatzkodierung
\usepackage[utf8]{inputenc}              % Einleseschriftkodierung
\usepackage{lmodern}                     % lmodern-Schriftsatz
\usepackage{microtype}                   % Microtypographie-Paket
\usepackage{babel}                       % Anpassung an deutschen Sprachraum
\usepackage{csquotes}
\usepackage{graphicx}
\usepackage{pdfpages}
\usepackage{multirow}
\usepackage{rotating}

\begin{document}
\begin{titlepage}
    \begin{center}
        \vspace*{3cm}
            
        \Huge
        \textbf{GitHub - Auswertung}
            
        \vspace{0.5cm}
        \LARGE
        Wissenschaftliches Arbeiten\\ 
        \Large
        Wintersemester 2022/2023
        \vspace{1.5cm}           
        \vfill    
        \vspace{0.8cm} 
        \Large
        Technische Universität Dortmund\\
            
    \end{center}
\end{titlepage}

\tableofcontents

\newpage
\section{Vorinformationen}
Wir haben einen Datensatz der von 100 Studierenden folgende Variablen erfasst:
\begin{itemize}
\item Alter
\item Studienfach
\item Interesse an Mathematik
\item Interesse an Programmieren
\item Mathe-LK
\end{itemize}

Da wir genau 100 Einträge im Datensatz haben, lässt sich an den relativen Häufigkeiten direkt die absolute Häufigkeit ablesen. Deshalb wird im folgenden meist nur eine der beiden Häufigkeiten angegeben.
\newpage
\section{Analyse der einzelnen Variablen}
\subsection{Alter}
Mittelwert: 25.02 \newline
Median:  25 \newline
Modus:  25 \newline
Standardabweichung:  1.588977 \newline
Kleinster Wert:  21 \newline
Größter Wert:  28 

\subsection{Studienfach}
\includegraphics[scale=0.8]{studienfach_barplot} \newline
Relative Häufigkeiten der verschiedenen Fächer: \\ \newline
\begin{tabular}{c|c|c|c}
Data Science & Informatik & Mathematik & Statistik \\
\hline
0.36 & 0.04 & 0.21 & 0.39 \\
\end{tabular}

\subsection{Interesse an Mathematik}
Die Studierenden gaben auf einer Skala von 1 bis 7 ihr Interesse an Mathematik an: \\ \newline
\begin{tabular}{c|c|c|c|c|c|c}
1 & 2 & 3 & 4 & 5 & 6 & 7 \\
\hline
8 & 7 & 6 & 7 & 34 & 19 & 19 \\
\end{tabular}
\newline \\
Nach Kategorisierung in niedriges, mittleres und hohes Interesse ergibt sich: \\ \newline
\begin{tabular}{c|c|c}
niedrig & mittel & hoch \\
\hline
21 & 60 & 19 \\
\end{tabular} 
\\


\subsection{Interesse an Informatik}
Die Studierenden gaben auf einer Skala von 1 bis 7 ihr Interesse an Informatik an: \\ \newline
\begin{tabular}{c|c|c|c|c|c|c}
1 & 2 & 3 & 4 & 5 & 6 & 7 \\
\hline
10 & 9 &15 &19& 13& 21& 13 \\
\end{tabular}
\newline \\
Nach Kategorisierung in niedriges, mittleres und hohes Interesse ergibt sich: \\ \newline
\begin{tabular}{c|c|c}
niedrig & mittel & hoch \\
\hline
19 & 68 & 13 \\
\end{tabular} 
\\

\begin{figure}[h]
   \begin{minipage}[b]{.4\linewidth}
      \includegraphics[scale=0.7]{interesse_mathe_barplot}
   \end{minipage}
   \hspace{.1\linewidth}% Abstand zwischen Bilder
   \begin{minipage}[b]{.4\linewidth} % [b] => Ausrichtung an \caption
	\includegraphics[scale=0.7]{interesse_info_barplot}
   \end{minipage}
\end{figure}

\subsection{Mathe-LK}
Von den 100 Studierenden hatten 26 keinen Mathe-LK und 74 hatten Mathe-LK. \\
\includegraphics[scale=0.7]{mathelk_barplot}

\newpage
\section{Zusammenhänge zwischen zwei Variablen}
\subsection{Alter und Mathe-LK}
Wir haben die Variable Alter aufgeteilt anhand der Information, ob sie Mathe-LK hatten oder nicht. \newline
\begin{tabular}{r|c|c}
& ohne Mathe-LK & mit Mathe-LK \\
\hline
Mittelwert: & 24.88462 & 25.06757 \\
Median:  & 25 & 25 \\
Modus: &  25 & 25\\
Standardabweichung: & 1.505375 & 1.624564\\
Kleinster Wert: & 22 & 21\\
Größter Wert: & 27 & 28\\
\end{tabular}

\newpage
\subsection{Studienfach und Interesse an Mathe}
\begin{tabular}{l||c|c|c|c|c|c|c}
& 1 & 2 & 3 & 4 & 5 & 6 & 7 \\
\hline
Data Science & 7 & 5 & 6 & 7 &11 & 0 & 0\\
Informatik  &  1 & 2 & 0 & 0 & 1&  0&  0\\
Mathe     &    0 & 0 & 0 & 0 & 8 & 6 & 7\\
Statistik     &0  &0 & 0 & 0 &14 &13& 12\\
\end{tabular}

Cramers Kontingenzindex:  0.5110956 \newline
Pearson Kontingenzindex:  0.6628377 


Nach der bereits bekannten Klassifizierung ergibt sich folgende Verteilung: \\ \newline
\begin{tabular}{l||c|c|c}
& niedrig & mittel & hoch \\
\hline
Data Science & 18 & 18 & 0\\
Informatik  & 3 & 1 & 0 \\
Mathe     &  0 & 14 & 7  \\
Statistik     &0 & 27 & 12\\
\end{tabular}

Cramers Kontingenzindex:  0.486885 \newline
Pearson Kontingenzindex:  0.5671212 

\includegraphics[scale=0.8]{studienfach_interesse_mathe_mosaikplot}

\subsection{Studienfach und Interesse an Informatik}
\begin{tabular}{l||c|c|c|c|c|c|c}
& 1 & 2 & 3 & 4 & 5 & 6 & 7 \\
\hline
Data Science & 0 & 0 & 0 & 0 & 5 &19 &12 \\
  Informatik   & 0 & 0 & 0 & 0 & 1 & 2 & 1\\
  Mathe        & 4 & 2 & 6 & 8 & 1 & 0 & 0\\
  Statistik    & 6 & 7 & 9 &11 & 6 & 0 & 0\\
\end{tabular}

Cramers Kontingenzindex:  0.5492657 \newline
Pearson Kontingenzindex:  0.6892657 


Nach der bereits bekannten Klassifizierung ergibt sich folgende Verteilung: \\ \newline
\begin{tabular}{l||c|c|c}
& niedrig & mittel & hoch \\
\hline
Data Science & 0 & 24 & 12\\
Informatik  & 0 & 3 & 1 \\
Mathe     &  6 & 15 & 0  \\
Statistik     &13 & 26 & 0\\
\end{tabular}

Cramers Kontingenzindex:  0.4037735 \newline
Pearson Kontingenzindex:  0.4958728 

\includegraphics[scale=0.8]{studienfach_interesse_info_mosaikplot}

\subsection{Interesse an Mathe und Interesse an Informatik}
Betrachte ab hier nur die klassifizierten Daten der Interesse an jeweils Mathe und Informatik. \\ \newline
\begin{tabular}{cc|c|c|c}
& \multicolumn{4}{c}{Interesse Info} \\
& & niedrig & mittel & hoch \\
& niedrig      & 0   &  16 &   5\\
&  mittel      & 13  &   39  &  8\\
&  hoch    &   6 &    13 &   0\\
 \begin{rotate}{90} Interesse Mathe \end{rotate}&&&&
\end{tabular}

Cramers Kontingenzindex:  0.2285451 \newline
Pearson Kontingenzindex:  0.3075465

\includegraphics[scale=0.9]{interesse_mathe_info_mosaikplot}

\subsection{Mathe-LK und Interesse an Mathe}
\begin{tabular}{c|c|c|c}
&  \multicolumn{3}{c}{Interesse an Mathe}   \\
&  niedrig & mittel & hoch\\
\hline
ohne Mathe-LK & 5  & 17 & 4  \\
 mit Mathe-LK &16 & 43 &15  \\
\end{tabular}

Cramers Kontingenzindex:  0.06810782 \newline
Pearson Kontingenzindex:  0.06795041

\includegraphics[scale=0.8]{mathelk_interesse_mathe_mosaikplot}

\newpage
\subsection{Mathe-LK und Interesse an Informatik}
\begin{tabular}{c|c|c|c}
&  \multicolumn{3}{c}{Interesse an Info}    \\
&  niedrig & mittel & hoch\\
\hline
 ohne Mathe-LK & 1 & 18 & 7\\
 mit Mathe-LK & 18 & 50 & 6\\
\end{tabular}

Cramers Kontingenzindex:  0.3081168 \newline
Pearson Kontingenzindex:  0.2944564 

\includegraphics[scale=0.8]{mathelk_interesse_info_mosaikplot}

\end{document}