\documentclass[paper=a4,                 % Papierformat
               fontsize=12pt,            % (Standard-) Schriftgröße
               parskip=half,             % Absatzformat halbzeilig
               ngerman,                 % deutsche Trennung
               ]{scrartcl}

\usepackage[T1]{fontenc}                 % Schriftsatzkodierung
\usepackage[utf8]{inputenc}              % Einleseschriftkodierung
\usepackage{lmodern}                     % lmodern-Schriftsatz
\usepackage{microtype}                   % Microtypographie-Paket
\usepackage{babel}                       % Anpassung an deutschen Sprachraum
\usepackage{csquotes}
\usepackage{graphicx}
\usepackage{pdfpages}
\usepackage{multirow}
\usepackage{rotating}
\usepackage[bottom =10mm]{geometry}

\begin{document}
\begin{titlepage}
    \begin{center}
        \vspace*{3cm}
            
        \Huge
        \textbf{GitHub - Auswertung}
            
        \vspace{0.5cm}
        \LARGE
        Wissenschaftliches Arbeiten\\ 
        \Large
        Wintersemester 2022/2023
        \vspace{2cm}      
           
        Lisa Harand 
        
		Jorrit Kühne
		
		Seyed Amir Homayoun Fazel 
		
		Wilson Wilson
		
		Ahmed Ghroubi 
		
		Eya Ghroubi 
        \vfill    
        \vspace{0.8cm} 
        \Large
        Technische Universität Dortmund\\
            
    \end{center}
\end{titlepage}

\tableofcontents

\newpage
% 1
\section{Vorinformationen}
Wir haben einen Datensatz der von 100 Studierenden folgende Variablen erfasst:
\begin{itemize}
\item Alter
\item Studienfach
\item Interesse an Mathematik
\item Interesse an Programmieren
\item Mathe-LK
\end{itemize}

Da wir genau 100 Einträge im Datensatz haben, lässt sich an den relativen Häufigkeiten 
direkt die absolute Häufigkeit ablesen. 
Deshalb wird im folgenden meist nur eine der beiden Häufigkeiten angegeben.
\newpage
% 2
\section{Analyse der einzelnen Variablen}
% 2.1
\subsection{Alter}
Mittelwert: 25.02 \newline
Median:  25 \newline
Modus:  25 \newline
Standardabweichung:  1.588977 \newline
Kleinster Wert:  21 \newline
Größter Wert:  28 

\newpage
%2.2
\subsection{Studienfach}
 \begin{figure}[h]
 	\begin{center}
 		\includegraphics[scale=0.7]{studienfach_barplot}
 	\end{center}
  \caption{absolute Häufigkeiten der Studienfächer}
 \end{figure} 
 

\begin{table}[h]
\begin{center}
\begin{tabular}{c|c|c|c}
Data Science & Informatik & Mathematik & Statistik \\
\hline
0.36 & 0.04 & 0.21 & 0.39 \\
\end{tabular}
\end{center}
\caption{relative Häufigkeiten der Studienfächer}
\end{table}

\newpage
%2.3
\subsection{Interesse an Mathematik und Informatik}
Die Studierenden gaben auf einer Skala von 1 bis 7 ihr Interesse an Mathematik und Informatik an: 

\begin{table}[h]
\begin{minipage}[b]{.5\linewidth}
	\begin{center}
\begin{tabular}{c||c|c|c|c|c|c|c}
Interesse & 1 & 2 & 3 & 4 & 5 & 6 & 7 \\
\hline
Anzahl & 8 & 7 & 6 & 7 & 34 & 19 & 19 \\
\end{tabular}
\end{center}
\caption{Interesse an Mathe}
\end{minipage}
\begin{minipage}[b]{.5\linewidth}
\begin{center}
\begin{tabular}{c||c|c|c|c|c|c|c}
Interesse & 1 & 2 & 3 & 4 & 5 & 6 & 7 \\
\hline
Anzahl & 10 & 9 &15 &19& 13& 21& 13 \\
\end{tabular}
\end{center}
\caption{Interesse an Informatik}
\end{minipage}
\end{table}

Nach Kategorisierung in niedriges, mittleres und hohes Interesse ergibt sich: 

\begin{table}[h]
\begin{minipage}[b]{.5\linewidth}
	\begin{center}
\begin{tabular}{c||c|c|c}
Interesse & niedrig & mittel & hoch \\
\hline
Anzahl & 21 & 60 & 19 \\
\end{tabular}
\end{center}
\caption{Interesse an Mathe \newline(klassifiziert)}
\end{minipage}
\begin{minipage}[b]{.5\linewidth}
\begin{center}
\begin{tabular}{c||c|c|c}
Interesse & niedrig & mittel & hoch \\
\hline
Anzahl & 19 & 68 & 13 \\
\end{tabular} 
\end{center}
\caption{Interesse an Informatik \newline(klassifiziert)}
\end{minipage}
\end{table}

\begin{figure}[h]
   \begin{minipage}[b]{.4\linewidth}
      \includegraphics[scale=0.7]{interesse_mathe_barplot}
      \caption{Interesse an Mathe (klassifiziert)}
   \end{minipage}
   \hspace{.1\linewidth}% Abstand zwischen Bilder
   \begin{minipage}[b]{.4\linewidth} % [b] => Ausrichtung an \caption
	\includegraphics[scale=0.7]{interesse_info_barplot}
	\caption{Interesse an Info  (klassifiziert)}
   \end{minipage}
\end{figure}

\newpage
%2.4
\subsection{Mathe-LK}
Von den 100 Studierenden hatten 26 keinen Mathe-LK und 74 hatten Mathe-LK.
\begin{figure}[h]
\begin{center}
\includegraphics[scale=0.7]{mathelk_barplot}
\caption{Angabe über Mathe-LK (0 $\widehat{=}$ hatte keinen Mathe-LK, 1 $\widehat{=}$ hatte Mathe-LK)}
\end{center}
\end{figure}

\newpage
% 3
\section{Zusammenhänge zwischen zwei Variablen}
%3.1
\subsection{Alter und Mathe-LK}
Wir haben die Variable Alter aufgeteilt anhand der Information, ob sie Mathe-LK hatten oder nicht. \newline
\begin{table}[h]
\begin{center}
\begin{tabular}{r|c|c}
& ohne Mathe-LK & mit Mathe-LK \\
\hline
Mittelwert: & 24.88462 & 25.06757 \\
Median:  & 25 & 25 \\
Modus: &  25 & 25\\
Standardabweichung: & 1.505375 & 1.624564\\
Kleinster Wert: & 22 & 21\\
Größter Wert: & 27 & 28\\
\end{tabular}
\caption{Lage- und Streuungsmaße des Alters anhand Mathe-LK}
\end{center}
\end{table}

\newpage
%3.2
\subsection{Studienfach und Interesse an Mathe}
\begin{table}[h]
\begin{center}
\begin{tabular}{l||c|c|c|c|c|c|c}
& 1 & 2 & 3 & 4 & 5 & 6 & 7 \\
\hline
Data Science & 7 & 5 & 6 & 7 &11 & 0 & 0\\
Informatik  &  1 & 2 & 0 & 0 & 1&  0&  0\\
Mathe     &    0 & 0 & 0 & 0 & 8 & 6 & 7\\
Statistik     &0  &0 & 0 & 0 &14 &13& 12\\
\end{tabular}
\caption{Interesse an Mathe aufgeteilt nach Studienfächern}
Cramers Kontingenzindex:  0.5110956, Pearson Kontingenzindex:  0.6628377
\end{center}
\end{table}
Nach der bereits bekannten Klassifizierung ergibt sich folgende Verteilung:
\begin{table}[h]
\begin{center}
\begin{tabular}{l||c|c|c}
& niedrig & mittel & hoch \\
\hline
Data Science & 18 & 18 & 0\\
Informatik  & 3 & 1 & 0 \\
Mathe     &  0 & 14 & 7  \\
Statistik     &0 & 27 & 12\\
\end{tabular}
\caption{Interesse an Mathe (klassifiziert) aufgeteilt nach Studienfächern}
Cramers Kontingenzindex:  0.486885, Pearson Kontingenzindex:  0.5671212 
\end{center}
\end{table}

\begin{figure}[ht]
   \begin{minipage}{.4\textwidth}
      \includegraphics[scale=0.5]{studienfach_interesse_mathe0_mosaikplot}
   \end{minipage}
   \hspace{.1\linewidth}% Abstand zwischen Bilder
   \begin{minipage}{.4\textwidth} % [b] => Ausrichtung an \caption
	\includegraphics[scale=0.5]{studienfach_interesse_mathe_mosaikplot}
   \end{minipage}
   \caption{Mosaikplots zu Tabellen 7 und 8}
\end{figure}

\newpage
%3.3
\subsection{Studienfach und Interesse an Informatik}
\begin{table}[h]
\begin{center}
\begin{tabular}{l||c|c|c|c|c|c|c}
& 1 & 2 & 3 & 4 & 5 & 6 & 7 \\
\hline
Data Science & 0 & 0 & 0 & 0 & 5 &19 &12 \\
  Informatik   & 0 & 0 & 0 & 0 & 1 & 2 & 1\\
  Mathe        & 4 & 2 & 6 & 8 & 1 & 0 & 0\\
  Statistik    & 6 & 7 & 9 &11 & 6 & 0 & 0\\
\end{tabular}
\caption{Interesse an Info aufgeteilt nach Studienfächern}
Cramers Kontingenzindex:  0.5492657, Pearson Kontingenzindex:  0.6892657 
\end{center}
\end{table}
Nach der bereits bekannten Klassifizierung ergibt sich folgende Verteilung:
\begin{table}[h]
\begin{center}
\begin{tabular}{l||c|c|c}
& niedrig & mittel & hoch \\
\hline
Data Science & 0 & 24 & 12\\
Informatik  & 0 & 3 & 1 \\
Mathe     &  6 & 15 & 0  \\
Statistik     &13 & 26 & 0\\
\end{tabular}
\caption{Interesse an Info (klassifiziert) aufgeteilt nach Studienfächern}
Cramers Kontingenzindex:  0.4037735, Pearson Kontingenzindex:  0.4958728
\end{center}
\end{table}

\begin{figure}[ht]
   \begin{minipage}{.4\textwidth}
      \includegraphics[scale=0.5]{studienfach_interesse_info0_mosaikplot}
   \end{minipage}
   \hspace{.1\linewidth}% Abstand zwischen Bilder
   \begin{minipage}{.4\textwidth} % [b] => Ausrichtung an \caption
	\includegraphics[scale=0.5]{studienfach_interesse_info_mosaikplot}
   \end{minipage}
   \caption{Mosaikplots zu Tabellen 9 und 10}
\end{figure}

\newpage
%3.4
\subsection{Interesse an Mathe und Interesse an Informatik}

\begin{table}[h]
\begin{center}
\begin{tabular}{c||c|c|c|c|c|c|c}
& 1 & 2 & 3 & 4 & 5 & 6 & 7 \\
\hline
  1 & 0& 0& 0& 0& 0& 5& 3\\
  2& 0& 0& 0& 0& 2& 4& 1\\
  3& 0& 0& 0& 0& 0& 5& 1\\
  4& 0& 0& 0& 0& 3& 2& 2\\
  5& 3& 4& 5& 7& 4& 5& 6\\
  6 &4 &2 &6 &4 &3 &0 &0\\
  7 &3& 3& 4& 8& 1& 0& 0\\
\end{tabular}
\caption{Interesse an Mathe (Zeilen) und Informatik (Spalten)}
Cramers Kontingenzindex:  0.3626015, Pearson Kontingenzindex:  0.6640713
\end{center}
\end{table}
Nach der bereits bekannten Klassifizierung ergibt sich folgende Verteilung:
\begin{table}[h]
\begin{center}
\begin{tabular}{c|c|c|c}
 & niedrig & mittel & hoch \\
 \hline
 niedrig      & 0   &  16 &   5\\
  mittel      & 13  &   39  &  8\\
  hoch    &   6 &    13 &   0\\
\end{tabular}
\caption{Interesse an Mathe (Zeilen) und Informatik (Spalten) (klassifiziert)}
Cramers Kontingenzindex:  0.2285451, Pearson Kontingenzindex:  0.3075465 
\end{center}
\end{table}

\begin{figure}[ht]
   \begin{minipage}{.4\textwidth}
      \includegraphics[scale=0.5]{interesse_mathe0_info0_mosaikplot}
   \end{minipage}
   \hspace{.1\linewidth}% Abstand zwischen Bilder
   \begin{minipage}{.4\textwidth} % [b] => Ausrichtung an \caption
	\includegraphics[scale=0.5]{interesse_mathe_info_mosaikplot}
   \end{minipage}
   \caption{Mosaikplots zu Tabellen 11 und 12}
\end{figure}

\newpage
%3.5
\subsection{Mathe-LK und Interesse an Mathe}
\begin{table}[h]
\begin{center}
\begin{tabular}{c|c|c|c}
&  \multicolumn{3}{c}{Interesse an Mathe}   \\
&  niedrig & mittel & hoch\\
\hline
ohne Mathe-LK & 5  & 17 & 4  \\
 mit Mathe-LK &16 & 43 &15  \\
\end{tabular}
\caption{Interesse an Mathe aufgeteilt nach Mathe-LK}
Cramers Kontingenzindex:  0.06810782, Pearson Kontingenzindex:  0.06795041
\end{center}
\end{table}

\begin{figure}[ht]
   \begin{minipage}{.4\textwidth}
      \includegraphics[scale=0.6]{mathelk_interesse_mathe0_mosaikplot}
   \end{minipage}
   \hspace{.1\linewidth}% Abstand zwischen Bilder
   \begin{minipage}{.4\textwidth} % [b] => Ausrichtung an \caption
	\includegraphics[scale=0.6]{mathelk_interesse_mathe_mosaikplot}
   \end{minipage}
   \caption{Mosaikplots zur Interesse an Mathe (unklassifiziert und klassifiziert)}
\end{figure}

\newpage
%3.6
\subsection{Mathe-LK und Interesse an Informatik}
\begin{table}[h]
\begin{center}
\begin{tabular}{c|c|c|c}
&  \multicolumn{3}{c}{Interesse an Info}    \\
&  niedrig & mittel & hoch\\
\hline
 ohne Mathe-LK & 1 & 18 & 7\\
 mit Mathe-LK & 18 & 50 & 6\\
\end{tabular}
\caption{Interesse an Mathe aufgeteilt nach Mathe-LK}
Cramers Kontingenzindex:  0.3081168, Pearson Kontingenzindex:  0.2944564 
\end{center}
\end{table}

\begin{figure}[ht]
   \begin{minipage}{.4\textwidth}
      \includegraphics[scale=0.6]{mathelk_interesse_info0_mosaikplot}
   \end{minipage}
   \hspace{.1\linewidth}% Abstand zwischen Bilder
   \begin{minipage}{.4\textwidth} % [b] => Ausrichtung an \caption
	\includegraphics[scale=0.6]{mathelk_interesse_info_mosaikplot}
   \end{minipage}
      \caption{Mosaikplots zur Interesse an Info (unklassifiziert und klassifiziert)}
\end{figure}

\end{document}